\chapter*{Введение}
\label{sec:afterwords}
\addcontentsline{toc}{chapter}{Введение}

\section*{Цель проекта}

Разработать и внедрить интегрированную систему для автоматизированного
отслеживания и представления статусов кабинетов НИЯУ МИФИ, путем создания
парсера для сбора актуальной информации о кабинетах с портала home.mephi и
разработки телеграмм-бота, который предоставляет удобный интерфейс
пользователям, исследующим наличие свободных аудиторий в реальном времени.

\section*{Функциональные требования}

\subsection*{a. Парсер:}
\begin{itemize}
    \item Автоматическое извлечение данных о кабинетах с портала home.mephi в
заданные интервалы времени.
    \item Сбор информации о расположении кабинетов, времени проведения занятий,
наличии проектора и другой необходимой информации.
    \item Обновление существующей базы данных новой информацией с сайта.
\end{itemize}

\subsection*{b. Телеграмм-бот:}
\begin{itemize}
    \item Интерактивный интерфейс для пользователей.
    \item Поиск свободных кабинетов в заданный период времени.
    \item Предоставление детализированной информации о кабинете (расположение,
наличие проектора и т. д.).
    \item Уведомления пользователю о статусе определенных кабинетов по запросу.
\end{itemize}

\section*{Инструментальные средства}

\begin{enumerate}
    \item \textbf{Draw.io:} Инструментальное средство для визуализации диаграмм,
предоставляющее возможности отображения архитектурных решений, процессов и
компонентного взаимодействия.
    \item \textbf{Golang (Go):} Эффективный язык программирования с
характеристиками высокой производительности, подходящий для разработки парсеров
в связи с способностью обрабатывать значительные объемы данных.
    \item \textbf{PostgreSQL на CleverCloud:} Облачная реализация базы данных
PostgreSQL обеспечивает устойчивость, защиту данных и масштабируемость,
минимизируя задачи администрирования.
    \item \textbf{Yandex Cloud:} Облачная инфраструктура, включающая в себя:
    \begin{enumerate}[\alph{enumii}.]
        \item \textbf{Yandex Cloud Functions:} Serverless-технология для
исполнения кода в ответ на события, исключающая необходимость серверного
администрирования.
        \item \textbf{Yandex API Gateway:} Средство управления входящими
API-запросами с функционалом безопасности и маршрутизации.
        \item \textbf{Yandex Message Queue:} Сервис для асинхронного обмена
сообщениями, подходящий для задач межкомпонентного взаимодействия.
    \end{enumerate}
    \item \textbf{IDE (Goland, Datagrip):} 
    \begin{itemize}
        \item Goland: Интегрированная среда разработки, оптимизированная для
языка Go.
        \item Datagrip: Инструмент для работы с базами данных, в т.ч.
PostgreSQL.
    \end{itemize}
    \item \textbf{GitHub:} Платформа для хранения и версионного контроля
исходного кода, предоставляющая инструменты для коллаборативной разработки.
\end{enumerate}