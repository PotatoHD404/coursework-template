\chapter*{Заключение}
\addcontentsline{toc}{chapter}{Заключение}

В ходе выполнения данной учебно-исследовательской работы было проведено
исследование инструментов развертывания контейнеризированных сред и возможностей
их интеграции с Terraform и Scala. Был проведен анализ системы Kubernetes и её
особенностей типизации, а также исследованы возможности интеграции системы
Kubernetes с Terraform. Было проведено сравнение различных систем, таких как
k3s, Kubernetes и Minikube.

В результате анализа была разработана алгебраическая модель для представления
определений Terraform, включая определение основных типов и структур данных, а
также функций для обработки и развертывания определений Terraform.

Была спроектирована архитектура модуля-обертки для развертывания типизированных
определений Terraform, включая разработку интерфейсов и классов модуля для
генерации HCL конфигурации, а также функций для взаимодействия с Kubernetes и
Terraform.

В ходе реализации был создан парсер для плагинов Terraform, парсер документации,
а также модуль Case Classes Generator. Был реализован модуль для работы с
Kubernetes API.

Однако, несмотря на значительные успехи, работа над проектом еще не завершена.
Некоторые из модулей, описанных в разделе 3, еще требуют реализации. Это
подчеркивает сложность и масштаб задачи, а также необходимость продолжения
работы над проектом.

В целом, проделанная работа демонстрирует возможность интеграции системы
Terraform с Scala и Kubernetes, что открывает новые возможности для
автоматизации и оптимизации процессов развертывания и масштабирования
приложений. Однако, для достижения полной функциональности и надежности,
необходимо продолжить разработку и тестирование оставшихся модулей.
